\documentclass[aspectratio=169, 14pt]{beamer}
\usepackage{listings}
\usepackage{xcolor}
\lstset { %
    language=C++,
    backgroundcolor=\color{black!5}, % set backgroundcolor
    basicstyle=\footnotesize,% basic font setting
}


\title{Using Pseudo-Random Number Repeatably in a Fine-Grain Multithreaded Simulation}
\subtitle{Second coding period: track-based reseeding of HepRandomEngine during simulation}
\author{Dmitry Savin}
\date{\today}

\usepackage[utf8]{inputenc}

\setbeamersize{text margin left=0pt,text margin right=0pt}

\begin{document}
\begin{large}

 \frame{\titlepage}

 \begin{frame}{G4Track}
  Added a field fHash to hold the hash to be used to set the generator.
  Chose int64\_t because:
  \begin{itemize}
   \item usually the same as G4long accepted as a seed by G4Random,
   \item standard fixed width - more predictable than (G4)long,
   \item small overhead,
   \item may be cast to intptr\_t to point to states for special engines.
  \end{itemize}
 \end{frame}

 \begin{frame}{G4TrackingManager}
  Added a flag fRngType to control if the hash values are used of enum type
  RngType with meaningful names:
  \begin{itemize}
   \item fDefault - unmodified geant4 behavior,
   \item fGeneric - setting the generators via base interface,
   \item fSpecial - setting generator using the imlementation details\\
    (if unimplemented switches to fGeneric).
  \end{itemize}
 \end{frame}

 \begin{frame}{G4TrackingMessenger}
  Added a G4UIcmdWithAString RNGCmd.\\
  It adds macro command "/tracking/rng" accepting one of three values:
  "default", "generic", "special" - 
  that naturally map to the G4TrackingManager flags.\\
  It is avaliable at the idle state of Geant4.
 \end{frame}

 \begin{frame}{G4TrackingManager::ProcessOneTrack}
  It is the natural place to introduce track dependent changes,
  because it is the topmost function that takes a single track.\\
  If the reseeding is activated:
  \begin{itemize}
   \item For the primary particle the hash is set by concatenating two int values from the RNG;
   \item Before tracking the hash is used as a seed for the RNG;
   \item After the tracking the hashes of the secondary particles are set
    by hasing the parent hash with the order in which they were created,
    which is detemined by the hash used as the seed.
  \end{itemize}
 \end{frame}
 
  
 \begin{frame}{Hashing function}
  Currently std::hash is used as the hashing function.\\
  To hash several values in one hash\_combine from boost is used.
  
%   template <class T> inline void hash_combine(std::size_t& seed, T const& v)
%   {
%     seed ^= std::hash<T>()(v) + 0x9e3779b9 + (seed << 6) + (seed >> 2);
%   }
  
  Its weakness should be compenstated by the RNG seeding function,
  but a check is needed, or a stronger hash.
 \end{frame}
 
 \begin{frame}{Test}
  The test is based on example B2b.\\\ \\
  A Stacking Action is added that can place a track to the waiting stack with $\frac{1}{2}$ probability.
  The behavior is controlled by a messenger.\\\ \\
  
  The output of the program is a histogram [of number of particles in each generation], which is saved in ROOT format.
 \end{frame} 

 \begin{frame}{Test macro}
  The test macro runs the simulation
  \begin{itemize}
   \item with several initial seeds,
   \item with default and random stacking order,
   \item with default and hash-based generator seeding
  \end{itemize}
  in all combinations.\\
  For each combination one hundred 1 GeV electrons are simulated.
 \end{frame} 
 
 \begin{frame}{Test post-processing}
  The resulting histograms are processed by a ROOT macro.\\
  The macro uses TH1::Chi2Test to calculate p-value for pairs of histograms
  from runs with the same seed or the same stacking and seeding.\\
  The p-value for equal histograms is 1, for different - less than 1.
 \end{frame}
 
 \begin{frame}{Test post-processing results}
  p-value for runs with default stacking and the same seeds is 1, with different seeds $<$ 1
  $\Rightarrow$ the test is sensitive to the random number generation.\\\ \\
  p-value for runs with default seeding and different stacking $<$ 1
  $\Rightarrow$ the test is sensitive to different stacking.\\\ \\
  p-value for runs with hash-based seeding and different stacking is 1
  $\Rightarrow$ hash-based seeding gives results independent from track processing order.
 \end{frame}
 
 \begin{frame}{Hashing algorithm}
  \scalebox{.5}{\setlength{\unitlength}{4144sp}%
%
\begingroup\makeatletter\ifx\SetFigFont\undefined%
\gdef\SetFigFont#1#2#3#4#5{%
  \reset@font\fontsize{#1}{#2pt}%
  \fontfamily{#3}\fontseries{#4}\fontshape{#5}%
  \selectfont}%
\fi\endgroup%
\begin{picture}(12708,5958)(397,-7315)
{\color[rgb]{0,0,0}\thicklines
\put(6751,-4178){\circle*{224}}
}%
{\color[rgb]{0,0,0}\put(6751,-6878){\circle*{224}}
}%
{\color[rgb]{0,0,0}\put(6751,-6203){\circle*{224}}
}%
{\color[rgb]{0,0,0}\put(6751,-5528){\circle*{224}}
}%
{\color[rgb]{0,0,0}\put(6751,-4853){\circle*{224}}
}%
{\color[rgb]{0,0,0}\put(6751,-3503){\circle*{224}}
}%
{\color[rgb]{0,0,0}\put(6751,-2828){\circle*{224}}
}%
\put(3181,-3612){\makebox(0,0)[lb]{\smash{{\SetFigFont{20}{24.0}{\rmdefault}{\bfdefault}{\updefault}{\color[rgb]{0,0,0}1}%
}}}}
\put(3061,-3496){\makebox(0,0)[lb]{\smash{{\SetFigFont{20}{24.0}{\rmdefault}{\bfdefault}{\updefault}{\color[rgb]{0,0,0}i}%
}}}}
\put(4861,-3496){\makebox(0,0)[lb]{\smash{{\SetFigFont{20}{24.0}{\rmdefault}{\bfdefault}{\updefault}{\color[rgb]{0,0,0}i}%
}}}}
\put(4981,-3612){\makebox(0,0)[lb]{\smash{{\SetFigFont{20}{24.0}{\rmdefault}{\bfdefault}{\updefault}{\color[rgb]{0,0,0}2}%
}}}}
\put(8461,-3496){\makebox(0,0)[lb]{\smash{{\SetFigFont{20}{24.0}{\rmdefault}{\bfdefault}{\updefault}{\color[rgb]{0,0,0}i}%
}}}}
\put(8581,-3612){\makebox(0,0)[lb]{\smash{{\SetFigFont{20}{24.0}{\rmdefault}{\bfdefault}{\updefault}{\color[rgb]{0,0,0}n}%
}}}}
{\color[rgb]{0,0,0}\put(11453,-4786){\oval(1574,900)}
}%
{\color[rgb]{0,0,0}\put(11431,-5911){\vector( 0, 1){720}}
}%
{\color[rgb]{0,0,0}\put(10666,-7261){\framebox(1575,1350){}}
}%
\put(10891,-6361){\makebox(0,0)[lb]{\smash{{\SetFigFont{20}{24.0}{\rmdefault}{\bfdefault}{\updefault}{\color[rgb]{0,0,0}hashed}%
}}}}
\put(10801,-6991){\makebox(0,0)[lb]{\smash{{\SetFigFont{20}{24.0}{\rmdefault}{\bfdefault}{\updefault}{\color[rgb]{0,0,0}pedigree}%
}}}}
\put(10801,-4921){\makebox(0,0)[lb]{\smash{{\SetFigFont{20}{24.0}{\rmdefault}{\bfdefault}{\updefault}{\color[rgb]{0,0,0}set seed}%
}}}}
{\color[rgb]{0,0,0}\put(4951,-6586){\oval(1350,1350)}
}%
{\color[rgb]{0,0,0}\put(4951,-4914){\oval(900,644)}
}%
{\color[rgb]{0,0,0}\put(8551,-4914){\oval(900,644)}
}%
{\color[rgb]{0,0,0}\put(8551,-6586){\oval(1350,1350)}
}%
{\color[rgb]{0,0,0}\put(3151,-6586){\oval(1350,1350)}
}%
{\color[rgb]{0,0,0}\put(3151,-4898){\oval(900,676)}
}%
{\color[rgb]{0,0,0}\put(1801,-6586){\vector( 1, 0){675}}
}%
{\color[rgb]{0,0,0}\put(3826,-6586){\vector( 1, 0){450}}
}%
{\color[rgb]{0,0,0}\put(6976,-6586){\vector( 1, 0){900}}
}%
{\color[rgb]{0,0,0}\put(9226,-6586){\vector( 1, 0){1440}}
}%
{\color[rgb]{0,0,0}\put(4951,-5279){\vector( 0,-1){643}}
}%
{\color[rgb]{0,0,0}\put(8551,-5279){\vector( 0,-1){643}}
}%
{\color[rgb]{0,0,0}\put(3151,-5279){\vector( 0,-1){643}}
}%
{\color[rgb]{0,0,0}\put(3151,-3886){\vector( 0,-1){675}}
}%
{\color[rgb]{0,0,0}\put(4951,-3886){\vector( 0,-1){675}}
}%
{\color[rgb]{0,0,0}\put(8551,-3886){\vector( 0,-1){675}}
}%
{\color[rgb]{0,0,0}\put(3151,-2311){\vector( 0,-1){675}}
}%
{\color[rgb]{0,0,0}\put(4951,-2311){\vector( 0,-1){675}}
}%
{\color[rgb]{0,0,0}\put(8551,-2311){\vector( 0,-1){675}}
}%
{\color[rgb]{0,0,0}\put(4501,-3843){\framebox(900,857){}}
}%
{\color[rgb]{0,0,0}\put(2701,-2268){\framebox(6300,857){}}
}%
{\color[rgb]{0,0,0}\put(8101,-3843){\framebox(900,857){}}
}%
{\color[rgb]{0,0,0}\put(2701,-3843){\framebox(900,857){}}
}%
{\color[rgb]{0,0,0}\put(451,-7261){\framebox(1350,1350){}}
}%
{\color[rgb]{0,0,0}\put(9901,-3886){\framebox(3150,2475){}}
}%
{\color[rgb]{0,0,0}\multiput(9901,-2536)(-344.11765,0.00000){26}{\line(-1, 0){172.059}}
\multiput(1126,-2536)(0.00000,-355.26316){10}{\line( 0,-1){177.632}}
\put(1126,-5911){\vector( 0,-1){0}}
}%
{\color[rgb]{0,0,0}\put(5626,-6586){\vector( 1, 0){900}}
}%
{\color[rgb]{0,0,0}\put(11431,-4291){\vector( 0, 1){450}}
}%
\put(541,-6361){\makebox(0,0)[lb]{\smash{{\SetFigFont{20}{24.0}{\rmdefault}{\bfdefault}{\updefault}{\color[rgb]{0,0,0}random}%
}}}}
\put(676,-7036){\makebox(0,0)[lb]{\smash{{\SetFigFont{20}{24.0}{\rmdefault}{\bfdefault}{\updefault}{\color[rgb]{0,0,0}seeds}%
}}}}
\put(2611,-6631){\makebox(0,0)[lb]{\smash{{\SetFigFont{14}{16.8}{\rmdefault}{\bfdefault}{\updefault}{\color[rgb]{0,0,0}combine}%
}}}}
\put(4411,-6631){\makebox(0,0)[lb]{\smash{{\SetFigFont{14}{16.8}{\rmdefault}{\bfdefault}{\updefault}{\color[rgb]{0,0,0}combine}%
}}}}
\put(8011,-6631){\makebox(0,0)[lb]{\smash{{\SetFigFont{14}{16.8}{\rmdefault}{\bfdefault}{\updefault}{\color[rgb]{0,0,0}combine}%
}}}}
\put(2881,-5011){\makebox(0,0)[lb]{\smash{{\SetFigFont{14}{16.8}{\rmdefault}{\bfdefault}{\updefault}{\color[rgb]{0,0,0}hash}%
}}}}
\put(4681,-5011){\makebox(0,0)[lb]{\smash{{\SetFigFont{14}{16.8}{\rmdefault}{\bfdefault}{\updefault}{\color[rgb]{0,0,0}hash}%
}}}}
\put(8281,-5011){\makebox(0,0)[lb]{\smash{{\SetFigFont{14}{16.8}{\rmdefault}{\bfdefault}{\updefault}{\color[rgb]{0,0,0}hash}%
}}}}
\put(4591,-1951){\makebox(0,0)[lb]{\smash{{\SetFigFont{34}{40.8}{\rmdefault}{\bfdefault}{\updefault}{\color[rgb]{0,0,0}pedigree}%
}}}}
\put(10351,-2311){\makebox(0,0)[lb]{\smash{{\SetFigFont{34}{40.8}{\rmdefault}{\bfdefault}{\updefault}{\color[rgb]{0,0,0}random}%
}}}}
\put(10126,-3436){\makebox(0,0)[lb]{\smash{{\SetFigFont{34}{40.8}{\rmdefault}{\bfdefault}{\updefault}{\color[rgb]{0,0,0}generator}%
}}}}
\put(10351,-2851){\makebox(0,0)[lb]{\smash{{\SetFigFont{34}{40.8}{\rmdefault}{\bfdefault}{\updefault}{\color[rgb]{0,0,0}number}%
}}}}
\put(1351,-3436){\makebox(0,0)[lb]{\smash{{\SetFigFont{20}{24.0}{\rmdefault}{\bfdefault}{\updefault}{\color[rgb]{0,0,0}number}%
}}}}
\put(1801,-3886){\makebox(0,0)[lb]{\smash{{\SetFigFont{20}{24.0}{\rmdefault}{\bfdefault}{\updefault}{\color[rgb]{0,0,0}of}%
}}}}
\put(1351,-4336){\makebox(0,0)[lb]{\smash{{\SetFigFont{20}{24.0}{\rmdefault}{\bfdefault}{\updefault}{\color[rgb]{0,0,0}secondary}%
}}}}
\put(946,-5776){\rotatebox{90.0}{\makebox(0,0)[lb]{\smash{{\SetFigFont{20}{24.0}{\rmdefault}{\bfdefault}{\updefault}{\color[rgb]{0,0,0}start event}%
}}}}}
\end{picture}%
}
 \end{frame}
 
 \end{large}
\end{document}
