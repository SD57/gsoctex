\documentclass[aspectratio=169, 14pt]{beamer}
\usepackage{listings}
\usepackage{xcolor}
\lstset { %
    language=C++,
    backgroundcolor=\color{black!5}, % set backgroundcolor
    basicstyle=\footnotesize,% basic font setting
}


\title{Using Pseudo-Random Number Repeatably in a Fine-Grain Multithreaded Simulation}
\subtitle{Second coding period: track-based reseeding of HepRandomEngine during simulation}
\author{Dmitry Savin}
\date{\today}

\usepackage[utf8]{inputenc}

\setbeamersize{text margin left=0pt,text margin right=0pt}

\begin{document}
\begin{large}

 \frame{\titlepage}

 \begin{frame}{G4Track}
  Added a field fHash to hold the hash to be used to set the generator.
  Chose int64\_t because:
  \begin{itemize}
   \item usually the same as G4long accepted as a seed by G4Random,
   \item standard fixed width - more predictable than (G4)long,
   \item small overhead,
   \item may be cast to intptr\_t to point to states for special engines.
  \end{itemize}
 \end{frame}

 \begin{frame}{G4TrackingManager}
  Added a flag fRngType to control if the hash values are used of enum type
  RngType with meaningful names:
  \begin{itemize}
   \item fDefault - unmodified geant4 behavior,
   \item fGeneric - setting the generators via base interface,
   \item fSpecial - setting generator using the imlementation details\\
    (if unimplemented switches to fGeneric).
  \end{itemize}
 \end{frame}

 \begin{frame}{G4TrackingMessenger}
  Added a G4UIcmdWithAString RNGCmd.\\
  It adds macro command "/tracking/rng" accepting one of three values:
  "default", "generic", "special" - 
  that naturally map to the G4TrackingManager flags.\\
  It is avaliable at the idle state of Geant4.
 \end{frame}

 \begin{frame}{G4TrackingManager::ProcessOneTrack}
  It is the natural place to introduce track dependent changes,
  because it is the topmost function that takes a single track.\\
  If the reseeding is activated:
  \begin{itemize}
   \item For the primary particle the hash is set by concatenating two int values from the RNG;
   \item Before tracking the hash is used as a seed for the RNG;
   \item After the tracking the hashes of the secondary particles are set
    by hasing the parent hash with the order in which they were created,
    which is detemined by the hash used as the seed.
  \end{itemize}
 \end{frame}
 
  
 \begin{frame}{Hashing function}
  Currently std::hash is used as the hashing function.\\
  To hash several values in one hash\_combine from boost is used.
  
%   template <class T> inline void hash_combine(std::size_t& seed, T const& v)
%   {
%     seed ^= std::hash<T>()(v) + 0x9e3779b9 + (seed << 6) + (seed >> 2);
%   }
  
  Its weakness should be compenstated by the RNG state transition function,
  but a check is needed.
 \end{frame}
 
 
 
 \end{large}
\end{document}
