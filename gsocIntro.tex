\documentclass{beamer}

\usepackage[utf8]{inputenc}
\usepackage{default}

\mode<presentation>{}

\title{Using Pseudo-random number repeatably in a fine-grain multithreaded simulation}
% \subtitle{The subtitle}
\author{Dmitry Savin}

\begin{document}
 \begin{frame}{About myself}
 \end{frame}

 \begin{frame}{Monte Carlo simulation}
 \end{frame}

 \begin{frame}{Random number generation}
 \end{frame}
 
 \begin{frame}{Reproducibility}
 \end{frame}
 
 \begin{frame}{Reproducibility and multithreading}
 \end{frame}
 
 \begin{frame}{Reproducibility and dynamical multithreading}
 \end{frame}

 \begin{frame}{Strands}
 \end{frame}

 \begin{frame}{Pedigrees}
 \end{frame}

 \begin{frame}{Types of generators}
 \end{frame}

 \begin{frame}{HepRandomEngine}
 \end{frame}

 \begin{frame}{SPRNG}
 \end{frame}

 \begin{frame}{Random123}
 \end{frame}
 
 \begin{frame}{PCG}
 \end{frame}
 
 \begin{frame}{DOTMIX}
 \end{frame}
 
 \begin{frame}{C++11 <random>}
 \end{frame}
 
 \begin{frame}{Geant4 workflow}
 \end{frame}

 \begin{frame}{GeantV workflow}
 \end{frame}
 
 \begin{frame}{Pedigrees for tracks}
 \end{frame}
 
 \begin{frame}{Pedigree prototype}
 \end{frame}
 
 \begin{frame}{Traversing order}
 \begin{itemize}
  \item Depth-first
  \item Width-first
 \end{itemize}

 \end{frame}
 
\end{document}
