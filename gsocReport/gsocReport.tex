\documentclass[a4paper, 12pt]{article} % a4 is 210mm x 297mm
\usepackage[dvipsnames]{xcolor}
\usepackage{listings}
\usepackage{graphicx}
\usepackage{tikz}
\usepackage{fancyvrb}
\usepackage{sectsty}
\lstset { %
    language=C++
}

\allsectionsfont{\centering}

% redefine \VerbatimInput
\RecustomVerbatimCommand{\VerbatimInput}{VerbatimInput}%
{fontsize=\footnotesize,
 %
 frame=lines,  % top and bottom rule only
 framesep=2em, % separation between frame and text
 rulecolor=\color{Gray},
 %
%  label=\fbox{\color{Black}data.txt},
 labelposition=topline,
 %
 commandchars=\|\(\), % escape character and argument delimiters for
                      % commands within the verbatim
 commentchar=*        % comment character
}

\title{Using Pseudo-Random Numbers Repeatably in a Fine-Grain Multithreaded Simulation}
\author{Dmitry Savin}
\date{\today}

\usepackage[utf8]{inputenc}

\newcommand{\MD}{Merkle-Damg\r{a}rd}

\begin{document}
 \maketitle

 \section*{ Motivation }
  Particle transport Monte Carlo simulations are a key tool for High Energy Physics experiments, including the LHC experiments at CERN.
  All Monte Carlo (MC) simulations depend vitally on Pseudo-Random Number Generators (PRNGs) used to sample many distributions.
  PRNGs used must possess very large periods, fast execution, the ability to create a large number of streams, and very good correlation properties between streams and sampled values. It must be possible to reproduce a simulated event any time, for reexamination or debugging.

  Because of the transition from event-level parallelism in Geant4 to dynamical multithreading in GeantV, the tracks in one event or even parts of the same track are processed by different threads.
  Thus to assure reproducibility the pseudo-random engine state must be associated with the track itself;
  and the state of the secondary track has to be a deterministic function of the parent track.
  
 \section*{ Goals }
  
 \section*{ Implementation }
 
  \subsection*{ Standalone prototype }
 
  \subsection*{ Geant4-based prototype }
  
 \section*{ Tests and benchmarks }
 
 \section*{ Conclusion }
 
 \section*{ Acknowledgements }
  Development sponsored by Google in Google Summer of Code 2017 under supervision of John Apostolakis and Sandro Wenzel.
 
 \section*{ Links }
 
%  \newpage
%  \bibliographystyle{plain}
%  \bibliography{gsocReport}
 
\end{document}
