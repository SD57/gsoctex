\documentclass[aspectratio=169, 14pt]{beamer}

\title{Using Pseudo-Random Number Repeatably in a Fine-Grain Multithreaded Simulation}
\author{Dmitry Savin}
\date{\today}

\usepackage[utf8]{inputenc}

\setbeamersize{text margin left=0pt,text margin right=0pt}

\begin{document}
\begin{large}

 \frame{\titlepage}

 \begin{frame}{About myself}
  \begin{itemize}
   \item PhD student at Dukhov Research Institute of Automatics
   \item Work on Geant4-based neutron transport simulation
   \item Graduated Moscow Institute of Physics and Technology
   \item Language of choice: C++
  \end{itemize}
 \end{frame}

 \begin{frame}{Motivation}
 \large
  \begin{itemize}
   \item Monte Carlo simulation is a key tool in High Energy Physics
   \item MC depends on Pseudo-Random Number Generators
   \item Half of LHC computing time is used for Geant4 simulation
   \item PRNGs take 1-5\% of CPU time in LHC simulations
    \begin{itemize}
     \item up to 10\% in neutron transport simulation 
    \end{itemize}
   \item Reproducibility from any point of simulation is needed
  \end{itemize}
 \end{frame}
 
  \begin{frame}{Motivation}
  \large
  \begin{itemize}
   \item Geant4 uses event-level parallelism
   \begin{itemize}
     \item Each event is simulated independently
     \item Independent PRNGs in a small fixed number of threads
    \end{itemize}
   \item GeantV uses track-level parallelism
    \begin{itemize}
     \item Similar tracks are collected in "baskets"
     \item Baskets are processed in parallel to utilize vectorisation
     \item A big number of tracks is needed to reduce overhead
     \item Number of tracks, baskets and threads changes during the simulation
    \end{itemize}

  \end{itemize}

 \end{frame}

 
 \begin{frame}{Solution}
  \begin{itemize}
   \item "Pedigrees" to assure reproducibility
    \begin{itemize}
     \item Each particle track holds the generator state
     \item The state depends on the parent track
     \item The state is independent from the processing thread
    \end{itemize}
   \item Counter-based PRNG for lightweight seeding
    \begin{itemize}
     \item The state is one or several integers without additional constraints
     \item Simple transition function (increment) allows to fast-forward
     \item The randomness is guaranteed by the output function (block cipher)
    \end{itemize}
  \end{itemize}

 \end{frame}
 
 \end{large}
\end{document}
