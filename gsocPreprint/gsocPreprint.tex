\documentclass[a4paper, titlepage, 12pt]{article} % a4 is 210mm x 297mm
\usepackage{listings}
\usepackage{xcolor}
\lstset { %
    language=C++
}


\title{Using Pseudo-Random Number Repeatably in a Fine-Grain Multithreaded Simulation}
\author{Dmitry Savin}
\date{\today}

\usepackage[utf8]{inputenc}

\newcommand{\MD}{Merkle-Damg\r{a}rd}

\begin{document}
 \maketitle
 
 \abstract
  Because of the transition from event-level parallelism in Geant4 to dynamical multithreading in GeantV the order in which tracks are processed becomes non-determenistic.
  Thus to maintain reproducibility one needs to associate the random generator state with the track itself and the worker thread currently processing the track.
  To be reproducible from the beginning of the event, this state can only depend on the pedigree of the track.
  To reduce overhead the pedigree is stored in a hashed form, which is sufficient because the reverse reconstruction is unneeded.
  Calculation of the hashed pedigree of a track using the hashed pedigree of the parent track makes the hashing algorithm a \MD\ construct with well-studied properties.
  
  We implement this construction using a 64-bit hash and standard hash with boost\_combine as the compression function operating on the number of the track among siblings as the input message blocks.
  We use the hashed pedigree as the seed for a CLHEP random number engine at the beginning of processing of each track in a Geant4 simulation.
  We show the reproducibility of the results under different track stacking order and their agrement with the results with the default random number generation.
  We show that the performance overhead is negligible for most of the random number generators except for those that have a big internal state.
  
 
 
\end{document}
